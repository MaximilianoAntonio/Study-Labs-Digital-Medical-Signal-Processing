\documentclass[10pt]{article}
% Formato extenso: report
\begin{sloppypar}
	
\end{sloppypar}
% Formato corto: article

% Esto es para que el LaTeX sepa que el texto está en español:
\usepackage[english]{babel}

\usepackage{amsmath, amsthm, amsfonts,amssymb}

% Bórrame si quieres:
\usepackage{multicol}

% Referencias
\usepackage{hyperref}

% Paquete para escribir código
\usepackage{listings}
\lstset{basicstyle=\footnotesize\ttfamily,breaklines=true}
\usepackage{alltt}

% Paquete para incluir imágenes
\usepackage{graphicx}

% Paquete para incluir varias imágenes en una
\usepackage{subfig}

% para poder fijar las imágenes ([H])
\usepackage{float}

% para agregar opciones al enumerate
\usepackage{enumerate}

% Teoremas
\newtheorem{thm}{Teorema}[section]
\newtheorem{cor}[thm]{Corolario}
\newtheorem{lem}[thm]{Lema}
\newtheorem{prop}[thm]{Proposición}
\theoremstyle{definition}
\newtheorem{defn}[thm]{Definición}
\theoremstyle{remark}
\newtheorem{rem}[thm]{Observación}
\theoremstyle{definition}
\newtheorem{prob}{Problema}
\numberwithin{equation}{prob}

% Calculus symbols
\newcommand{\pd}[2]{\frac{\partial #1}{ \partial #2}}   % First partial derivative command
\newcommand{\td}[2]{\frac{\mathrm{d} #1}{ \mathrm{d} #2}}
\newcommand{\pdd}[2]{\frac{\partial^2 #1}{ \partial #2 ^2}}   % Second partial derivative command
\newcommand{\pddc}[3]{\frac{\partial^2 #1}{ \partial #2 \partial #3}}   % Second partial derivative command

% Continuum mechanics & FEM symbols
\def\sca   #1{\mbox{\rm{#1}}{}}
\def\mat   #1{\mbox{\boldmath $\mathsf #1$}}
\def\vec   #1{\mbox{\boldmath $#1$}{}}
\def\ten   #1{\mbox{\boldmath $#1$}{}}
\def\ltr   #1{\mbox{\sf{#1}}}
\def\bltr  #1{\mbox{\sffamily{\bfseries{{#1}}}}}

% math operators and symbols
\DeclareMathOperator{\dive}{div}
\DeclareMathOperator{\trace}{trace}
\DeclareMathOperator{\tr}{tr}
\DeclareMathOperator{\symm}{symm}
\DeclareMathOperator{\sk}{skew}
\DeclareMathOperator{\grad}{grad}
\DeclareMathOperator{\Grad}{Grad}
\DeclareMathOperator{\curl}{curl}
\DeclareMathOperator{\Curl}{Curl}
\def\R{\mbox{\(\mathbb{R}\)}}
\def\dx{\mbox{\(\,\mathrm{d}x\)}}


\usepackage{geometry}
\geometry{left=2.5cm, right=2.5cm, top=2cm, bottom=3cm}

\usepackage{makeidx}
\makeindex


\begin{document}
	
	\begin{titlepage}
		
		
		%%%%% NO MODIFICAR
		\begin{figure}
			\begin{minipage}{4cm}
				\includegraphics[width=0.9\textwidth]{./figures/logo}
			\end{minipage}
			\begin{minipage}{11cm}
				\vspace{4mm}
				{\sc UNIVERSIDAD DE VALPARAÍSO}\\
				Escuela de Ingeniería Civil Biomédica\\
				{\bf CBM414 Procesamiento digital de señales biomédicas}\\
				\vspace{0mm}
				\hrulefill
			\end{minipage}
		\end{figure}
		\phantom{""}\vspace{60mm}
		
		
		%%%%% MODIFICAR
		\begin{center}
			\Huge{\textbf{Bonificación 2 Evaluación 2}}\vspace{95mm}\\
			\raggedleft \Large{Maximiliano Antonio Gaete Pizarro}\\ 
		\end{center}
		
		
	\end{titlepage}
	
\printindex

\section*{Ejemplo 3.4.7}

Este ejemplo trata de encontrar la ecuación en diferencias I/O (entrada-salida) de un sistema basado en un filtro IIR causal, dado por su respuesta al impulso \( h(n) \). También se busca obtener la ecuación en diferencias para la respuesta al impulso misma.

\subsection*{Definición de \( h(n) \)}

La respuesta al impulso \( h(n) \) está definida de la siguiente manera:

\[
h(n) = 
\begin{cases} 
2, & \text{si } n = 0 \\
4(0.5)^{n-1}, & \text{si } n \geq 1 
\end{cases}
\]

Lo que significa que \( h(0) = 2 \) y luego, para \( n \geq 1 \), \( h(n) \) decae exponencialmente con una tasa de \( 0.5 \).

\subsection*{Relación entre \( y(n) \) y \( x(n) \)}

La salida \( y(n) \) se puede expresar como la convolución entre la entrada \( x(n) \) y la respuesta al impulso \( h(n) \):

\[
y(n) = \sum_{k=0}^{n} h(k) x(n-k)
\]

Sustituyendo los valores numéricos de \( h(n) \), tenemos:

\[
y_n = h_0 x_n + h_1 x_{n-1} + h_2 x_{n-2} + h_3 x_{n-3} + \cdots
\]
\[
y_n = 2x_n + 4x_{n-1} + 2 \left( x_{n-2} + 0.5x_{n-3} + 0.5^2 x_{n-4} + \cdots \right)
\]

Para el valor de salida anterior \( y_{n-1} \), tenemos:

\[
y_{n-1} = 2x_{n-1} + 4x_{n-2} + 2 \left( x_{n-3} + 0.5x_{n-4} + \cdots \right)
\]

\subsection*{Transformación a ecuación en diferencias}

Multiplicamos la ecuación para \( y_{n-1} \) por 0.5:

\[
0.5 y_{n-1} = x_{n-1} + 2 \left( x_{n-2} + 0.5 x_{n-3} + 0.5^2 x_{n-4} + \cdots \right)
\]

Restando esta ecuación de la ecuación original para \( y_n \), obtenemos:

\[
y_n - 0.5 y_{n-1} = 2x_n + 3x_{n-1}
\]

Por lo tanto, la ecuación en diferencias I/O es:

\[
y(n) = 0.5 y(n-1) + 2x(n) + 3x(n-1)
\]

\subsection*{Ecuación en diferencias para \( h(n) \)}

Si sustituimos \( x(n) = \delta(n) \) (el impulso unitario), obtenemos la ecuación en diferencias para \( h(n) \) de la siguiente manera:

\[
h(n) = 0.5 h(n-1) + 2\delta(n) + 3\delta(n-1)
\]

Esta ecuación es consistente con la secuencia de valores para \( h(n) \) que habíamos definido al inicio, demostrando que el análisis es correcto.


\section*{Ejemplo 3.4.8}

Este ejemplo nos pide encontrar la forma convolucional y la respuesta al impulso causal de un filtro IIR, dada por la siguiente ecuación en diferencias:

\[
y(n) = 0.25y(n-2) + x(n)
\]

\subsection*{Respuesta al impulso \( h(n) \)}

La respuesta al impulso \( h(n) \) debe satisfacer la misma ecuación en diferencias que \( y(n) \), dado que \( h(n) \) es la salida cuando la entrada es un impulso unitario \( \delta(n) \). Así, tenemos:

\[
h(n) = 0.25h(n-2) + \delta(n)
\]

La condición inicial es que \( h(n) = 0 \) para \( n < 0 \), es decir:

\[
h(-1) = h(-2) = 0
\]

Ahora, calculamos los primeros valores de \( h(n) \) mediante iteración:

\[
h(0) = 0.25h(-2) + \delta(0) = 0.25 \cdot 0 + 1 = 1
\]
\[
h(1) = 0.25h(-1) + \delta(1) = 0.25 \cdot 0 + 0 = 0
\]
\[
h(2) = 0.25h(0) + \delta(2) = 0.25 \cdot 1 + 0 = 0.25 = (0.5)^2
\]
\[
h(3) = 0.25h(1) + \delta(3) = 0.25 \cdot 0 + 0 = 0
\]
\[
h(4) = 0.25h(2) + \delta(4) = 0.25 \cdot 0.25 + 0 = 0.0625 = (0.5)^4
\]
\[
h(5) = 0.25h(3) + \delta(5) = 0.25 \cdot 0 + 0 = 0
\]

Y así sucesivamente. Como podemos observar, los valores de \( h(n) \) para \( n \geq 0 \) se alternan entre cero y potencias de \( 0.5 \) en los índices pares.

\subsection*{Expresión general de \( h(n) \)}

De este análisis, podemos deducir la forma general de \( h(n) \):

\[
h(n) = 
\begin{cases} 
(0.5)^n, & \text{si } n \text{ es par} \\
0, & \text{si } n \text{ es impar}
\end{cases}
\]

Equivalente a escribirlo como:

\[
h = [1, 0, (0.5)^2, 0, (0.5)^4, 0, (0.5)^6, 0, \dots]
\]

\subsection*{Forma convolucional}

Finalmente, la ecuación en diferencias que describe la salida \( y(n) \) en función de la entrada \( x(n) \), se puede escribir en forma convolucional. La ecuación general de convolución se define como:

\[
y(n) = \sum_{k=0}^{n} h(k)x(n-k)
\]

Sustituyendo los valores de \( h(n) \) obtenidos previamente, la ecuación convolucional queda:

\[
y_n = x_n + 0.5^2 x_{n-2} + 0.5^4 x_{n-4} + 0.5^6 x_{n-6} + \dots
\]

Esta es la forma convolucional de la ecuación en diferencias, que es equivalente a la ecuación original dada.
	



\end{document}